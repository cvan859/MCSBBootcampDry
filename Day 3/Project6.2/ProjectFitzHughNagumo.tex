\documentclass{exam}
\usepackage[utf8]{inputenc}

\usepackage[margin=1in]{geometry}
\usepackage{amsmath,amssymb}
\usepackage{multicol}
\usepackage{enumerate}
\usepackage{graphicx}
\usepackage[version=3]{mhchem}
\usepackage{listings}
\usepackage{color}
\definecolor{dkgreen}{rgb}{0,0.6,0}
\definecolor{gray}{rgb}{0.5,0.5,0.5}
\definecolor{mauve}{rgb}{0.58,0,0.82}

\lstset{frame=tb,
  language=Matlab,
  aboveskip=3mm,
  belowskip=3mm,
  showstringspaces=false,
  columns=flexible,
  basicstyle={\small\ttfamily},
  numbers=none,
  numberstyle=\tiny\color{gray},
  keywordstyle=\color{blue},
  commentstyle=\color{dkgreen},
  stringstyle=\color{mauve},
  breaklines=true,
  breakatwhitespace=true,
  tabsize=3
}

\setlength{\parindent}{0.0in}
\setlength{\parskip}{0.05in}

%\include{preamble}

%\renewcommand{\thesection}{{Part \arabic{section}}}

% Header and footer
\pagestyle{headandfoot}
\header{UCI MCSB Bootcamp Dry (Mathematical/Computational)}{}{}
\headrule
%\footer{\it{jun.allard@uci.edu}}{}{Page \thepage\ of \numpages}
\footrule
%%%%%%%%%%%%%%%%%%%%%%%%%%%%%%%%%%%%%%%%%%%%%%%%%%%%%%%%%
\begin{document}


%%%%%%%%%%%%%%%%%%%%%%%%%%%%%%%%%%%%%%%%%%%%%%%%%%%%%%%%%
\section*{Project 6.2: FitzHugh-Nagumo}
%%%%%%%%%%%%%%%%%%%%%%%%%%%%%%%%%%%%%%%%%%%%%%%%%%%%%%%%%
 
 
 The phenomenon of excitability exists in many biological systems, including in the electrophysiology of neurons. The FitzHugh-Nagumo equations
 \begin{align}
 \frac{d v}{dt} &= v - \frac{1}{3} v^3 - w \\
 \frac{d w}{dt} &= \epsilon \left( v + a - bw\right)
 \end{align}
 describe neuron electrophysiology where, roughly speaking, $v$ is the electrical potential (voltage) across the cell's membrane, and $w$ is the activity of ion pumps. The parameters $\epsilon$, $a$ and $b$ represent properties of the ion pumps. The model has been nondimensionalized. Both $v$ and $w$ can be negative or positive.
 
 
\begin{enumerate}
\item Confirm that for $\epsilon=0.08$, $a=0.5$, $b=0.2$, the system is oscillatory.
\item Confirm that for $\epsilon=0.08$, $a=1.0$, $b=0.2$. the system is excitable. Specifically, if you choose initial conditions $v(0)=-1.5, w(0)=-0.5$, the system evolves directly towards a stable steady state, but if you choose initial conditions $v(0)=-0.0, w(0)=-0.5$, the system moves away from the steady state, before eventually converging towards the steady state.
\end{enumerate}

Assume the neuron is at rest (at its steady state), and another cell injects a current into it. The current is injected between $t=40$ and $t=47$, and has a strength of $I_0=1.0$. In the model, this means 
 \begin{align}
 \frac{d v}{dt} &= v - \frac{1}{3} v^3 - w +I(t)\\
 \frac{d w}{dt} &= \epsilon \left( v + a - bw\right) 
 \end{align}
 where 
 \begin{equation}
 I(t) = \begin{cases}
 I_0 & \qquad t_{\rm start}<t < t_{\rm stop}\\
 0 & \qquad \mbox{otherwise}
 \end{cases}
 \end{equation}
or, in Matlab,
\begin{lstlisting}
I0 = 1.0;
tStart = 40;
tStop = 47;
I =@(t) I0*(t>tStart).*(t<tStop);
\end{lstlisting}

\begin{enumerate}
\setcounter{enumi}{2}
\item At the excitable parameters from above ($a=1.0$), simulate the system with initial conditions at the steady state (or very close), and simulate an injection at $t=40$ as above.
\end{enumerate}


Neurons are connected in a neural network. Suppose there are ten cells, each with membrane potential and ion pump activity obeying the FitzHugh-Nagumo equations for $v_i(t)$ and $w_i(t)$ where $i=1..10$ indexes the cells. The cells are electrically connected so that 
 \begin{align}
 \frac{d v_i}{dt} &= v_i - \frac{1}{3} v_i^3 - w_i +I_i(t) + D\left( v_{i-1} - 2 v_i + v_{i+1}\right)\\
 \frac{d w_i}{dt} &= \epsilon \left( v_i + a - bw_i\right) 
 \end{align}
where $D=0.9$ is a new parameter that described the electrical connectivity of the neighboring cells. The ion pumps are not connected between cells, so the $w$ equation is unchanged. For simplicity, let's assume the cells are connected in a ring, so that 
\begin{equation}
 \frac{d v_1}{dt} = v_1 - \frac{1}{3} v_1^3 - w_1 +I_1(t) + D\left( v_{10} - 2 v_1 + v_{2}\right)\\
\end{equation}
and similarly for the tenth cell. 
\begin{enumerate}
\setcounter{enumi}{3}
\item Write Matlab code to simulate these ten cells. There will be 20 equations $v,w$ for each cell. 
\begin{enumerate}
\item Assume there is no injection current. We expect all ten cells to settle at the same steady state. Make two plots. First, plot a time series of the membrane potential of all ten cells as a function of time. Second, make a movie of the voltage in all ten cells, where the horizontal axis is the cell number. 
\begin{lstlisting}
% movie
for nt=1:numel(T)
    figure(5); clf; hold on; box on;
    plot(X(nt,1:10)); 
    set(gca,'ylim', [-2.5,2.5])
    xlabel('Cell');
    ylabel('Voltage')
    drawnow;
end
\end{lstlisting}

\item Now assume that the fourth cell (and only the fourth cell) receives an injection current $I(t)$ as above, between $t=40$ and $t=47$. Make a time series with all ten cells. Make a movie with voltage for all cells.~\footnote{This type of behavior  is called an \emph{excitable traveling wave pulse}. These are unlike harmonic pulses familiar from sound waves, vibrations, light. For example, when two excitable pulses collide, they annihilate.}

Hint: The most elegant and extensible code design choice would be to write the functions for a general number of cells $N_{\rm cell}$. If you cannot figure out how to do that, in this case, it is possible to make it work with an inelegant solution of writing out all 20 ODEs. Make a judicious choice between elegance and getting things done.
\end{enumerate}
\end{enumerate}





  
%%%%%%%%%%%%%%%%%%%%%%%%%%%%%%%%%%%%%%%%%%%%%%%%%%%%%%%%%
\end{document}
%%%%%%%%%%%%%%%%%%%%%%%%%%%%%%%%%%%%%%%%%%%%%%%%%%%%%%%%%





In this problem, we ask, if the axon is subject to a force, which mechanical element ``takes a bigger hit"?
%%%%%%%%%%
\begin{figure}[h!]
\centering\includegraphics[width=10cm]{figHW11}
\end{figure}
%%%%%%%%%%


